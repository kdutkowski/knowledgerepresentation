\documentclass{mini}
\usepackage[utf8]{inputenc}
\usepackage{caption}
\usepackage{subcaption}
\usepackage[polish]{babel}
\usepackage{graphicx}
\usepackage{mathtools}
\usepackage{algpseudocode}
\usepackage{color}
\usepackage{xcolor}
\usepackage{listings}
\usepackage{catchfilebetweentags}
\usepackage{enumitem}

\usepackage{catchfilebetweentags}
\usepackage{etoolbox}
\makeatletter
\patchcmd{\CatchFBT@Fin@l}{\endlinechar\m@ne}{}
  {}{\typeout{Unsuccessful patch!}}
\makeatother

\addto\extraspolish{%  
 \def\figureautorefname{Rysunek}%  
} 

%------------------------------------------------------------------------------%
\title{Realizacje scenariuszy działań}
\author{Robert Jakubowski - Szef
\\Mariusz Ambroziak
\\Paweł Bielicki
\\Karol Bocian
\\Hanna Dziegciar
\\Karol Dzitkowski
\\Mateusz Jankowski
\\Wiktor Ryciuk}
\monthyear{\today}
%------------------------------------------------------------------------------%
\begin{document}
%<*tag>
\section{Przykłady}

\subsection{Pytanie czy dany scenariusz moze wystąpić}

\subsubsection{Historia}

Michał jest pracującym  studentem. W środę o godzinie 8.00 powinien  pojawić się w pracy zupełnie trzeźwy. Mimo to we wtorek postanowił pójść do baru. Jeśli Michał się napije,  stanie się pijany. Jeśli pójdzie spać przestanie być pijany, ale stanie się skacowany, co również będzie niedopuszczalne w jego pracy.

\subsubsection{Opis akcji}

 \textit{(drink, 2)}  \textbf{ causes}  \textit{drunk}\\
\textit{(sleep, 8)}   \textbf{causes} \textit{$ !$drunk}\\
\textit{(sleep, 8)}   \textbf{causes} \textit{hangover}  \textbf{if}  \textit{drunk}\\


\subsubsection{Scenariusze}
$Sc$ =$(OBS,ACS)$\\
$OBS$ =  $ {\{( ! drunk,\ 0),\ \ ( ! hangover,\ 0),\ 
( ! drunk,\ 10),\ \ ( ! hangover,\ 10)\}}$\\
$ACS$ = ${\{((drink,\ 2),\ 0),\ \ ((sleep,\ 8),\ 2)\}}$\\


$Sc_2$ =$(OBS_2,\ ACS_2)$\\
$OBS_2$ =  $ {\{( ! drunk,\ 0),\ \ ( ! hangover,\ 0),\ 
( ! drunk,\ 10),\ \ ( ! hangover,\ 10)\}}$\\
$ACS_2$ = ${\{((sleep,\ 8),\ 1)\}}$\\


\subsubsection{Kwerendy}

\begin{enumerate}
	\item  \textbf{ever executable} \textit{Sc}
	\item  \textbf{ever executable} \textit{Sc2}
\end{enumerate}

\subsubsection{Analiza}

Odpowiedzi na kwerendy to odpowiednio:
\begin{enumerate}
	\item \texttt{FALSE} zgodnie z diagramem \ref{PicSC1}.
	\item \texttt{TRUE} zgodnie z diagramem \ref{PicSC2}.
\end{enumerate}

%Zgodnie z diagramem dla scenariusza  \textit{Sc}
\begin{figure}[h!]
	\centering
	\includegraphics[width=1\textwidth]{Example1}
	\caption{Diagram dla scenariusza $Sc$}
	\label{PicSC1}
\end{figure}
%\begin{center}
%  \includegraphics[width=1\textwidth]{Example1}
%\end{center}

%Zgodnie z diagramem dla scenariusza  \textit{Sc2}

\begin{figure}[h!]
	\centering
	\includegraphics[width=1\textwidth]{Example1a}
	\caption{Diagram dla scenariusza $Sc_2$}
	\label{PicSC2}
\end{figure}
%\begin{center}
%  \includegraphics[width=1\textwidth]{Example1a}
%\end{center}

Scenariusz \textit{ Sc2} jest w pełni poprawny i wykonywalny. Scenariusza \textit{Sc} nie można wykonać, ponieważ wymaga on by w chwili $10$ fluenty \textit{drunk} i \textit{hangover} miały wartość FALSE, jednak w tej chwili zmienna \textit{hangover} ma wartość TRUE.


\subsection{Pytanie czy dany warunek zachodzi w danym czasie}

\subsubsection{Historia}

Mick i Sarah są parą, więc mają wspólne produkty spożywcze, ale posiłki zwykle jadają oddzielnie. Pewnego dnia Sarah chce zrobić ciasto, a Mick naleśniki. Nie mogą być one robione w tym samym czasie ze względu konieczność użycia miksera do przygotowania obu. Ponadto, zrobienie jednego lub~drugiego dania zużywa cały zapas jajek dostępnych w mieszkaniu, więc trzeba je potem dokupić.

\subsubsection{Opis akcji}
$(making\_panc,\ 1)$   \textbf{causes} $ !$\textit{eggs} \textbf{ if } \textit{eggs}\\
$(making\_cake,\ 1)$   \textbf{causes} $ !$\textit{eggs} \textbf{ if } \textit{eggs}\\
$(buy\_eggs,\ 2)$  \textbf{causes}  \textit{eggs}\\


\subsubsection{Scenariusz}

 $Sc$ =$(OBS,\ ACS)$\\
 $OBS$ =  ${ \{(eggs,\ 0)\} }$\\
 $ACS$ = $\{((making\_panc,\ 1),\ 0),\ \ ((making\_cake,\ 1),\ 2)\}$


\subsubsection{Kwerendy}

\begin{enumerate}
	\item  \textit{eggs}  \texttt{at} 0  \textbf{when}  \textit{Sc}
	\item  \textit{eggs} \texttt{at} 2 \textbf{ when} \textit{ Sc}
\end{enumerate}

\subsubsection{Analiza}

Odpowiedzi na kwerendy to odpowiednio:
\begin{enumerate}
	\item \texttt{TRUE},
	\item \texttt{FALSE}.
\end{enumerate}
	
Zgodnie z diagramem dla scenariusza \textit{Sc}:

\begin{figure}[h!]
	\centering
	\includegraphics[width=1\textwidth]{Example2}
	\caption{Diagram dla scenariusza $Sc$}
\end{figure}
%\begin{center}
%  \includegraphics[width=1\textwidth]{Example2}
%\end{center}

	Oczywiście warunek akcji \textit{making\_panc} nie jest spełniony w momencie $2$.
	
\newpage
\subsection{Pytanie czy dana akcja jest wykonywana w pewnym czasie}


\subsubsection{Historia}

Mamy Billa i psa Maxa. Jeśli Bill idzie, to Max biegnie przez jakiś czas. Jeśli Bill gwiżdże, Max szczeka przez jakiś czas. Jeśli Bill zatrzymuje się, Max również. Jeśli Bill przestaje gwizdać, to Max przestaje szczekać.

\subsubsection{Opis akcji}

$(goes\_Bill,\ 2)$ \textbf{causes} $run\_Max$\\
$(goes\_Bill,\ 2)$ \textbf{invokes} $(runs\_Max,\ 2)$ \textbf{after} $0$\\
$(runs\_Max,\ 2)$ \textbf{causes} $ ! run\_Max$\\
$(whistles\_Bill,\ 1)$ \textbf{causes} $bark\_Max$\\
$(whistles\_Bill,\ 1)$ \textbf{invokes} $(barks\_Max,\ 1)$ \textbf{after} $0$\\
$(barks\_Max,\ 1)$ \textbf{causes} $ ! bark\_Max$\\


\subsubsection{Scenariusz}

$Sc$ =$(OBS,\ ACS)$\\
$OBS$ = $\{ ( ! run\_Max, 0), (  ! bark\_Max, 0)  \}$\\
$ACS$ = $\{ ((goes\_Bill,\ 2),\ 1),\ \ ((whistles\_Bill,\ 1),5),\ \ ((goes\_Bill,\ 2),\ 7)\}$\\

\subsubsection{Kwerendy}

\begin{enumerate}
\item \textbf{performing} $runs\_Max$ \textbf{at} $8$ \textbf{when} \textit{Sc}
\item \textbf{performing} $runs\_Max$ \textbf{when} \textit{Sc}
\item \textbf{performing} \textbf{at} $8$ \textbf{when} \textit{Sc}
\end{enumerate}

\subsubsection{Analiza}
Odpowiedzi na powyższe kwerendy są następujące:
\begin{enumerate}
\item \texttt{FALSE},
\item \texttt{TRUE},
\item \texttt{TRUE}.
\end{enumerate}
Ilustruje to poniższy diagram:

\begin{figure}[h!]
	\centering
	\includegraphics[width=1\textwidth]{Example3}
	\caption{Diagram dla scenariusza $Sc$}
\end{figure}
%\begin{center}
%  \includegraphics[width=1\textwidth]{Example3}
%\end{center}
\newpage
\subsection{Pytanie czy podany cel jest osiągalny}


\subsubsection{Historia}
Mamy Billa oraz komputer. Bill może nacisnąć przycisk \textit{Włącz} lub odłączyć komputer od zasilania. Początkowo komputer jest wyłączony i podłączony do zasilania. Jeżeli zostanie naciśnięty jego przycisk \textit{Włącz} oraz komputer jest podłączony do zasilania, to komputer włączy się. Odłączenie komputera od prądu powoduje, że komputer będzie odłączony od~zasilania oraz wyłączony. 

\subsubsection{Opis akcji}

$(clicks\_button\_on,\ 1)$ \textbf{invokes} $(switches\_on\_computer,\ 2)$ \textbf{after} $0$ \textbf{if} $connect\_power\_computer$\\
$(switches\_on\_computer,\ 2)$ \textbf{causes} $on\_computer$\\
$(disconnects\_power,\ 1)$ \textbf{causes} $ ! connect\_power\_computer$\\
$(disconnects\_power,\ 1)$ \textbf{causes} $ ! on\_computer$\\

\subsubsection{Scenariusz}

\textit{Sc} =$(OBS,\ ACS)$\\
\textit{OBS} =  $\{ ( ! on\_computer, 0), (connect\_power\_computer, 0) \}$\\
\textit{ACS} = $\{(clicks\_button\_on,\ 1),\ 1),\ \ ((disconnects\_power,\ 1),\ 4),\ \ ((clicks\_button\_on,\ 1),\ 5)\}	$\\

\subsubsection{Kwerendy}

\begin{enumerate}
\item $on\_computer$ \textbf{at} $7$ \textbf{when} \textit{Sc}
\item \textbf{accesible} $on\_computer$ \textbf{when} \textit{Sc}
\end{enumerate}

\newpage
\subsubsection{Analiza}
Odpowiedzi na powyższe kwerendy są następujące:
\begin{enumerate}
\item \texttt{FALSE},
\item \texttt{TRUE}.
\end{enumerate}

Należy dodać, że  cel osiągnięto już w chwili czasu równej $4$, mimo iż komputer jest wyłączony od chwili czasu równej $5$ i stan ten nie ulega już zmianie.

Ilustruje to poniższy diagram:

\begin{figure}[h!]
	\centering
	\includegraphics[width=1\textwidth]{Example5}
	\caption{Diagram dla scenariusza $Sc$}
\end{figure}
%\begin{center}
%  \includegraphics[width=1\textwidth]{Example5}
%\end{center}



%</tag>
\end{document}
