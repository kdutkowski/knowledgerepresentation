\documentclass{mini}
\usepackage[utf8]{inputenc}
\usepackage{caption}
\usepackage{subcaption}
\usepackage[polish]{babel}
\usepackage{graphicx}
\usepackage{mathtools}
\usepackage{algpseudocode}
\usepackage{color}
\usepackage{xcolor}
\usepackage{listings}
\usepackage{catchfilebetweentags}
\usepackage{enumitem}

\usepackage{catchfilebetweentags}
\usepackage{etoolbox}
\makeatletter
\patchcmd{\CatchFBT@Fin@l}{\endlinechar\m@ne}{}
  {}{\typeout{Unsuccessful patch!}}
\makeatother

\addto\extraspolish{%  
 \def\figureautorefname{Rysunek}%  
} 

%------------------------------------------------------------------------------%
\title{Realizacje scenariuszy działań}
\author{Robert Jakubowski - Szef
\\Mariusz Ambroziak
\\Paweł Bielicki
\\Karol Bocian
\\Hanna Dziegciar
\\Karol Dzitkowski
\\Mateusz Jankowski
\\Wiktor Ryciuk}
\monthyear{\today}
%------------------------------------------------------------------------------%
\begin{document}
%<*tag>

\chapter{Opis języka kwerend}

Zdefiniowany język akcji może być odpytywany przez poniżej zaprezentowany język kwerend, który zapewnia uzyskanie odpowiedzi $TRUE/FALSE$ na następujące pytania:

Q1. Czy podany scenariusz jest możliwy do realizacji zawsze/kiedykolwiek?
\begin{itemize}
	\item $always/ever\ executable\ Sc$\\
	Oznacza, że scenariusz $Sc$ zawsze/kiedykolwiek jest możliwy do realizacji.
\end{itemize}

Q2. Czy w chwili $t\ge0$ realizacji podanego scenariusza warunek $\gamma$ zachodzi zawsze/kiedykolwiek?
\begin{itemize}
	\item $always/ever\ \gamma\ at\ t\ when\ Sc$\\
	Oznacza, że zawsze/kiedykolwiek w chwili $t$ realizacji scenariusza $Sc$ zachodzi warunek $\gamma$.
\end{itemize}

Q3. Czy w chwili $t$ realizacji scenariusza wykonywana jest akcja $A$?
\begin{itemize}
	\item $performing\ A\ at\ t\ when\ Sc$\\
	Oznacza, że zawsze w chwili $t$ realizacji scenariusza $Sc$ zachodzi akcja $A$.
\end{itemize}

Q4. Czy podany cel $\gamma$ jest osiągalny zawsze/kiedykolwiek przy zadanym zbiorze obserwacji OBS?
\begin{itemize}
	\item $always/ever\ accesible\ \gamma\ when\ Sc$\\
	Oznacza, że cel $\gamma$ jest osiągalny zawsze/kiedykolwiek przy zadanym zbiorze obserwacji $OBS$ przy realizacji scenariusza $Sc$.
\end{itemize}

Semantyka kwerend w języku\\
Niech $Sc$ będzie scenariuszem, $D$ niech będzie opisem domeny języka, wtedy powiemy, że kwerenda $Q$ jest konsekwencją $Sc$ zgodnie z $D$ (ozn. $Sc,\ D\ |\approx\ Q $)

\begin{itemize}
	\item zapytanie kwerendą $Q$ postaci $\gamma\ at\ t\ when\ Sc$\\ zwróci wynik $TRUE$ jeśli dla każdego modelu $S=(H,O,E)$ scenariusza $Sc$ zgodnego z $D$ zajdzie $H(\gamma,t)=1$
	\item zapytanie kwerendą $Q$ postaci $performing\ A\ at\ t\ when\ Sc$\\ zwróci wynik $TRUE$ jeśli dla każdego modelu $S=(H,O,E)$ scenariusza $Sc$ zgodnego z $D$ zajdzie $(A,t) \in E$
	\item zapytanie kwerendą $Q$ postaci $accesible\ \gamma\ when\ Sc$\\ zwróci wynik $TRUE$ jeśli dla każdego modelu $S=(H,O,E)$ scenariusza $Sc$ zgodnego z $D$ zajdzie $\exists_{t \in NN} \exists_{A \in Ac}\ \gamma \in O(A, t)$
\end{itemize}
jeśli warunek nie zajdzie program zwróci wartość $FALSE$.

%</tag>
\end{document}
