\documentclass{mini}
\usepackage[utf8]{inputenc}
\usepackage{caption}
\usepackage{subcaption}
\usepackage[polish]{babel}
\usepackage{graphicx}
\usepackage{mathtools}
\usepackage{algpseudocode}
\usepackage{color}
\usepackage{xcolor}
\usepackage{listings}
\usepackage{catchfilebetweentags}
\usepackage{enumitem}

\usepackage{catchfilebetweentags}
\usepackage{etoolbox}
\makeatletter
\patchcmd{\CatchFBT@Fin@l}{\endlinechar\m@ne}{}
  {}{\typeout{Unsuccessful patch!}}
\makeatother

\addto\extraspolish{%  
 \def\figureautorefname{Rysunek}%  
} 

%------------------------------------------------------------------------------%
\title{Realizacje scenariuszy działań}
\author{Robert Jakubowski - Szef
\\Mariusz Ambroziak
\\Paweł Bielicki
\\Karol Bocian
\\Hanna Dziegciar
\\Karol Dzitkowski
\\Mateusz Jankowski
\\Wiktor Ryciuk}
\monthyear{\today}
%------------------------------------------------------------------------------%
\begin{document}
%<*tag>
\section{Przykłady}


\subsection{Pytanie czy dana akcja jest wykonywana w pewnym czasie}

Ten przykład pokazuje przypadek kwerendy, która pyta, czy dana akcja jest wykonywana w pewnym czasie.

	\subsubsection{Historia}

Mamy Billa i psa Maxa. Jeśli Bill idzie, to  Max biegnie. Jeśli Bill gwiżdże , Max szczeka. Jeśli Bill zatrzymuje się, Max również. Jeśli Bill przestaje gwizdać, to Max przestaje szczekać.

	\subsubsection{Opis akcji}

\textbf{initially} $\neg go\_Bill$   \textbf{and}  $\neg run\_Max$  \textbf{and}  $\neg whistle\_Bill$  \textbf{and}  $\neg bark\_Max$ \\
$(goes\_Bill,2)$  \textbf{causes} $running\_Max$\\
$(goes\_Bill,2)$  \textbf{invokes} $(run\_Max,2)$  \textbf{after} $1$\\
$(whistles\_Bill,1)$  \textbf{causes} $barking\_Max$\\
$(whistles\_Bill,1)$  \textbf{invokes} $(barks\_Max,1)$  \textbf{after} $1$\\


	\subsubsection{Scenariusz}

\textit{Sc} =$(OBS,ACS)$\\
\textit{OBS} = ${\emptyset}$\\
\textit{ACS}  = ${ 	(goes\_Bill,0+1), (whistles\_Bill,5+2),(goes\_Bill,7+2)	}$\\

	\subsubsection{Kwerendy}
	
\begin{enumerate}
	\item \textbf{performing} $running\_Max$  \textbf{at} $8$ \textbf{when} \textit{Sc}
	\item \textbf{performing} $running\_Max$   \textbf{when} \textit{Sc}
	\item \textbf{performing}   \textbf{at} $8$ \textbf{when} \textit{Sc}
\end{enumerate}

	\subsubsection{Analiza}
Odpowiedzi na powyższe kwerendy są następujące:
\begin{enumerate}
	\item \texttt{FALSE},
	\item \texttt{TRUE},
	\item \texttt{TRUE}.
\end{enumerate}
Ilustruje to poniższy diagram:

\begin{center}
  \includegraphics[width=1\textwidth]{Example3}
\end{center}

\subsection{Brak integralności}

Przykład \textit{Brak integralnośći} pokazuje scenariusz, który mimo zgodności z warunkami zadania, jest sprzeczny z logiką \textit{common sense} (z powodu braku warunków integralności).

\subsubsection{Historia}
Mamy Billa oraz komputer. Bill może nacisnąć przycisk \textit{Włącz} lub odłączyć komputer od zasilania. Komputer jest wyłączony i podłączony do zasilania. Jeżeli zostanie naciśnięty jego przycisk \textit{Włącz}, to komputer włącza się.

\subsubsection{Opis akcji}

\textbf{initially} $\neg on\_computer$   \textbf{and}  $connects\_power\_computer$ \textbf{and} $\neg swithing\_on\_computer$\\
$(click\_button\_on,1)$  \textbf{causes} $switching\_on\_computer$\\
$(click\_button\_on,1)$  \textbf{invokes} $(switch\_on\_computer,2)$  \textbf{after} $1$\\
$(switch\_on\_computer,1)$  \textbf{causes} $on\_computer$\\
$(disconnect\_power,1)$  \textbf{causes} $on\_computer$  \textbf{and} $\neg swithing\_on\_computer$ \\

\subsubsection{Scenariusz}

\textit{Sc} =$(OBS,ACS)$\\
\textit{OBS} = ${\emptyset}$\\
\textit{ACS}  = ${ 	(click\_button\_on,0+1), (disconnect\_power,3+1),(click\_button\_on,4+1)	}$\\

\subsubsection{Kwerendy}

\begin{enumerate}
	\item $swithing\_on\_computer$  \textbf{at} $6+2$ \textbf{when} \textit{Sc}
	\item $swithing\_on\_computer$   \textbf{and}  $\neg on\_computer$ \textbf{at} $6+2$ \textit{when} \textbf{Sc}
\end{enumerate}


\subsubsection{Analiza}
Powyższy scenariusz jest prawidłowy, lecz zawiera pewną niezgodność. W chwili $t = 4+1$ komputer zostaje odcięty od zasilania. Powinien więc wyłączyć się.  Bill chwili $t = 5+1$ naciska przycisk \textit{Włącz}.Komputer zacznie włączać się mimo iż jest odcięty od zasilania. Zachodzą dwa sprzeczne ze sobą stany, tj.  $swithing\_on\_computer = T$  i $on\_computer=T$. Odpowiedzi na powyższe kwerendy będą odpowiednio: $1.$ \texttt{TRUE} i $2.$ \texttt{FALSE}. Należy zaznaczyć, że odpowiedzi zgodnie z logiką \textit{commonsense} powinny być sobie równe.

\begin{center}
  \includegraphics[width=1\textwidth]{Example5}
\end{center}

%</tag>
\end{document}
